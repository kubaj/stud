\documentclass[11pt,a4paper]{article}
\usepackage[slovak]{babel}
\usepackage[utf8]{inputenc}
\usepackage[left=2cm,text={17cm, 24cm},top=3cm]{geometry}
\usepackage{times}
\usepackage{ragged2e}
\newcommand{\myuv}[1]{\quotedblbase #1\textquotedblleft}
\bibliographystyle{czplain}

\begin{document}

\begin{titlepage}
\begin{center}
\Huge
\textsc{
    Vysoké učení technické v~Brně \\ [-.25em]
    \huge Fakulta informačních technologií}\\
\vspace{\stretch{0.382}}
\LARGE
Typografie a publikování -- 4. projekt \\ [-.25em]
\vspace{\stretch{0.618}}

{\Large \today \hfill
Jakub Kulich}
\end{center}
\end{titlepage}

\section{Typografia}
Typografia je umelecký odbor, ktorý sa zaoberá grafickou úpravou tlačených dokumentov. Toto zahŕňa všetko od dizajnu písma, cez usporiadanie jednotlivých znakov a odsekov až po zalamovanie textu.\cite{kniha2}


Typografia má tiež svoju históriu ako aj ostatné druhy umenia.\cite{article:his} História typografie sa odvíjala na základe: 
\begin{itemize}
    \item Časového obdobia 
    \item Územia (typografia na západe sa vyvíjala iným smerom ako typografia na východe)
\end{itemize}
Najväčší pokrok v typografii nastal v zlome medzi 20. a 21. storočím s príchodom digitalizácie.\cite{article:hist}


\section{\LaTeX}
\LaTeX~je typografický systém na sádzanie dokumentov pre profesionálne použitie.\cite{kniha1} Od klasických typografických systémov ako je Microsoft Office Word sa líši spôsobom vytvárania dokumentu. 
Tvorenie dokumentov v systéme \LaTeX~je podobné programovaniu stránok pomocou značiek v jazyku HTML -- HyperText Markup Language.\cite{html} 


\LaTeX~je vhodný na vytváranie diplomových prác\cite{diplomka}, bakalárských prác\cite{bakalarka} a kníh. Okrem toho je veľmi obľúbený pre zápis matematických výrazov. Zápis matematických výrazov je jednoduchší ako v bežných editoroch. Ešte väčší rozdiel je vo výsledku, kde v kvalite vysadenia týchto výrazov vyhráva \LaTeX.

\section{Grafický dizajn} 
Grafický dizajn je odvetvie kde sa riešia grafické problémy pomocou grafických prvkov a textu. Cieľom grafického dizajnu je zaujať pozorovateľa. Okrem estetiky musí grafický dizajn dbať aj na funkčnosť.\cite{htmlgrafika}


Aby sa mohol stať človek dizajnérom, musí mať v sebe štipku talentu a kreativity. Dennou prácou dizajnéra je tvoriť logá pre rôzne firmy, obaly pre výrobky všetkých druhov, atď.\cite{asdf}


Pre grafický dizajn sa väčšinou využívajú programy pre vektorovú grafiku ako je napríklad Inkscape.\cite{grsw}
\newpage
\raggedright
\bibliography{lit}

\end{document}
