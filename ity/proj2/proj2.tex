\documentclass[11pt,a4paper,twocolumn]{article}
\usepackage[czech]{babel}
\usepackage[utf8]{inputenc}
\usepackage[left=1.5cm,text={18cm, 25cm},top=2.5cm]{geometry}
\usepackage{amsthm}
\usepackage{amsmath}
\usepackage{amsfonts}
\usepackage{latexsym}
\usepackage{times}

\newcommand{\myuv}[1]{\quotedblbase #1\textquotedblleft}

\theoremstyle{definition}
\newtheorem{defn}{Definice}[section]
\newtheorem{algo}[defn]{Algoritmus}
\newtheorem{veta}{Věta}
\begin{document}

\begin{titlepage}
\begin{center}
\Huge
\textsc{
    Fakulta informačních technologií \\ [-.35em]
Vysoké učení technické v~Brně} \\
\vspace{\stretch{0.382}}
\LARGE
Typografie a publikování -- 2. projekt \\ [-.25em]
Sazba dokumentů a matematických výrazů
\vspace{\stretch{0.618}}

\end{center}
{\Large 2016 \hfill
Jakub Kulich}
\end{titlepage}

\section*{Úvod}
V~této úloze si vyzkoušíme sazbu titulní strany, matematických vzorců, prostředí a dalších textových struktur obvyklých pro technicky zaměřené texty, například rovnice (\ref{eq:1}) nebo definice \ref{def:1} na straně \pageref{def:1}. 

Na titulní straně je využito sázení nadpisu podle optického středu s~využitím zlatého řezu. Tento postup byl probírán na přednášce.


\section{Matematický text}
Nejprve se podíváme na sázení matematických symbolů a výrazů v~plynulém textu. Pro množinu $V$ označuje card($V$) kardinalitu $V$.
Pro množinu $V$ reprezentuje $V^*$ volný monoid generovaný množinou $V$ operací konkatenace.
Prvek identity ve volném monoidu $V^*$ značíme symbolem $\varepsilon$.
Nechť $V^+ = V^* - \{\varepsilon\}$. Algebraicky je tedy $V^+$ volná pologrupa generovaná množinou $V$ s~operací konkatenace.
Konečnou neprázdnou množinu $V$ nazvěme \textit{abeceda}.
Pro $w \in V^*$ označuje $|w|$ délku řetězce $w$. Pro $W \subseteq V$ označuje occur$(w,W)$ počet výskytů symbolů z~$W$ v~řetězci $w$ a sym$(w,i)$ určuje $i$-tý symbol řetězce $w$; například sym$(abcd,3) = c$. 

Nyní zkusíme sazbu definic a vět s~využitím balíku \texttt{amsthm}.

\begin{defn}\label{def:1}\textit{Bezkontextová gramatika} je čtveřice $G = (VTPS)$, kde $V$ je totální abeceda,
    $T \subseteq V$ je abe\-ceda terminálů, $S \in (V - T)$ je startující symbol a $P$ je konečná množina \textit{pravidel}
    tvaru $q: A \rightarrow \alpha$, kde $A \in (V-T)$, $\alpha \in V^*$ a $q$ je návěští tohoto pravidla. Nechť $N = V -T$ značí abecedu neterminálů.
Pokud $q: A \rightarrow \alpha$, $\gamma$ , $\delta \in V^*$, $G$ provádí derivační krok z~$\gamma A \delta$ do $\gamma \alpha \delta$ podle pravidla $q: A \rightarrow \alpha$, symbolicky píšeme 
$\gamma A \delta \Rightarrow \gamma \alpha \delta$ $[q: A \rightarrow \alpha]$ nebo zjednodušeně $\gamma A \delta \Rightarrow \gamma \alpha \delta$. Standardním způsobem definujeme $\Rightarrow ^m$, kde $m \geq 0$ . Dále definujeme 
tranzitivní uzávěr $\Rightarrow ^+$ a tranzitivně-reflexivní uzávěr $\Rightarrow ^*$ .
\end{defn}
Algoritmus můžeme uvádět podobně jako definice textově, nebo využít pseudokódu vysázeného ve vhodném prostředí (například \texttt{algorithm2e}).

\begin{algo}\textit{
Algoritmus pro ověření bezkontextovosti gramatiky. Mějme gramatiku $G = (N, T, P, S)$.
\begin{enumerate}
\item Pro každé pravidlo $p \in P$ proveď test, zda $p$ na levé straně obsahuje právě jeden symbol z~$N$ .
\item Pokud všechna pravidla splňují podmínku z~kroku $1$ tak je gramatika $G$ bezkontextová.
\end{enumerate}
}\end{algo}
\begin{defn}Jazyk definovaný gramatikou $G$ definujeme jako $L(G) = \left\{ w \in T^*|S \Rightarrow ^* w \right\}$.\end{defn}

\subsection{Podsekce obsahující větu}

\begin{defn}Nechť $L$ je libovolný jazyk. $L$ \textit{je bezkontextový jazyk}, když a jen když $L = L(G)$, kde $G$ je libovolná bezkontextová gramatika.\end{defn}

\begin{defn}Množinu $\mathcal{L} _{CF} = \{L|L$ je bezkontextový jazyk$\}$ nazýváme \textit{třídou bezkontextových jazyků}.\end{defn}

\begin{veta}\label{veta:1}Nechť $L_{abc} = \left\{ a^nb^nc^n|n \geq 0 \right\}$. Platí, že $L_{abc} \notin \mathcal{L} _{CF}$.\end{veta}

\begin{proof}Důkaz se provede pomocí Pumping lemma pro bezkontextové jazyky, kdy ukážeme, že není možné, aby platilo, což bude implikovat pravdivost věty \ref{veta:1}.\end{proof}

\section{Rovnice a odkazy}

Složitější matematické formulace sázíme mimo plynulý text. Lze umístit několik výrazů na jeden řádek, ale pak je třeba tyto vhodně oddělit, například příkazem \verb|\quad|.

$$\sqrt[x^2]{y^3_0} \quad \mathbb{N} = \left\{0,1,2,...\right\} \quad x^{y^y} \ne x^{yy} \quad z_{i_j} \not \equiv z_{ij} $$ 

V~rovnici (\ref{eq:1}) jsou využity tři typy závorek s~různou explicitně definovanou velikostí.

\begin{eqnarray} \label{eq:1}
    \bigg\{ \Big[ \left( a+b\right)*c \Big]^d + 1\bigg\} &=& x \\
    \lim_{x\to\infty} \frac{\sin^2x + \cos^2x}{4} &=& y \nonumber
\end{eqnarray}


V~této větě vidíme, jak vypadá implicitní vysázení limity $\lim_{x\to\infty}f(n)$ v~normálním odstavci textu. Podobně je to i s~dalšími symboly jako $\sum_{1}^{n}$ či $\bigcup _{A \in \mathcal{B}}$ . V~případě vzorce $\lim\limits_{x\to 0} \frac{\sin x}{x}$ jsme si vynutili méně úspornou sazbu příkazem \verb|\limits|.

\begin{eqnarray} 
    \int\limits_{a}^{b}f\left(x\right)\mathrm{d}x &=& - \int_{b}^{a} f\left(x\right)\mathrm{d}x \\
    \left(\sqrt[5]{x^4}\right)' = \left(x^{\frac{4}{5}}\right)' &=& \frac{4}{5}x^{-\frac{1}{5}} = \frac{4}{5\sqrt[5]{x}} \\
    \overline{\overline{A \lor B}} &=& \overline{\overline{A} \land \overline{B}}
\end{eqnarray}

\section{Matice}

Pro sázení matic se velmi často používá prostředí \texttt{array} a závorky (\verb|\left|, \verb|\right|). 

$$\left(
\begin{array}{cc}
    a+b & b -a \\
    \widehat{\xi + \omega} & \hat{\pi} \\
    \vec{a} & \stackrel{\leftarrow\!\rightarrow}{AC} \\
    0 & \beta
\end{array}
\right)$$

$$\textbf{A} = \left\|
\begin{array}{cccc}
    a_{11} & a_{12} & \hdots & a_{1n} \\
    a_{21} & a_{22} & \hdots & a_{2n} \\
    \vdots & \vdots & \ddots & \vdots \\
    a_{m1} & a_{m2} & \hdots & a_{mn} \\
\end{array} \right\|$$
$$
\left|\begin{array}{cc}
    t & u \\
    v & w
\end{array}\right|
= tw - uv $$
Prostředí \texttt{array} lze úspěšně využít i jinde.

$$ \binom{n}{k} = \left\{ 
\begin{array}{l l} 
    \frac{n!}{k!(n-k)!} & \text{pro } 0 \leq k \leq n \\
    0 & \text{pro } k < 0  \text{ nebo } k > n
\end{array}\right.$$


\section{Závěrem}

V~případě, že budete potřebovat vyjádřit matematickou konstrukci nebo symbol a nebude se Vám dařit jej nalézt v~samotném LaTeXu, doporučuji prostudovat možnosti balíku maker \AmS-\LaTeX.
\end{document}
